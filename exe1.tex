%%%%%%%%%%%%%%%%%%%%%%%%%%%%%%%%%%%%%%%%%%%%%%%%%%%%%%%%%%%%%%%%%%%%%%%%%%%%%%%%
% Author : [Name] [Surname], Tomas Polasek (template)
% Description : First exercise in the Introduction to Game Development course.
%   It deals with an analysis of a selected title from the point of its genre, 
%   style, and mechanics.
%%%%%%%%%%%%%%%%%%%%%%%%%%%%%%%%%%%%%%%%%%%%%%%%%%%%%%%%%%%%%%%%%%%%%%%%%%%%%%%%

\documentclass[a4paper,10pt,english]{article}

\usepackage[left=2.50cm,right=2.50cm,top=1.50cm,bottom=2.50cm]{geometry}
\usepackage[utf8]{inputenc}
\usepackage{hyperref}
\hypersetup{colorlinks=true, urlcolor=blue}

\newcommand{\ph}[1]{\textit{[#1]}}

\title{%
Analysis of Mechanics%
}
\author{%
Petr Šebela (xsebelp00)%
}
\date{}

\begin{document}

\maketitle
\thispagestyle{empty}

{%
\large

\begin{itemize}

\item[] \textbf{Title:} Factorio

\item[] \textbf{Released:} 2020 [ early access since 2016 ]

\item[] \textbf{Author:} Wube Software LTD.

\item[] \textbf{Primary Genre:} Automation

\item[] \textbf{Secondary Genre:} Sandbox

\item[] \textbf{Style:} Prerendered

\end{itemize}

}

\section*{\centering Analysis}
Factorio is a game about automation. Although automation principles have been used in games before (such as Minecraft`s redstone system), Factorio was the first to incorporate it as a core game-play mechanic. 
\newline
\newline
Factorio's gameplay is quite simple. You build a factory to research new things, then you use them to speed up, optimize and scale up the research process. This is the core gameplay loop of Factorio. It seems quite simple, but if you dig just a little deeper you will find some interesting systems that break up the loop and make the game less repetitive. For example, if you scale up your factory too fast, you create a lot of pollution. Pollution attracts and levels up Biters (enemies in Factorio), who will attempt to destroy your factory. More pollution means stronger enemies, so if you neglected defense of your factory, you are in a lot of trouble. 
\newline
\newline
Not all subsystems in Factorio are there to go against the player. For example, there are many of them that will actually help you achieve your goal of scaling your the factory. There is a whole railway system, that you can use to transport a lot of things over long distances. Another example of a useful subsystem is the logistic system. It enables you to set up transport of items without the need to physically connect the two sites together. Last but not least, is a logical network. It helps you manage your factory according to specific conditions. For example, if fluid in certain tank drops below certain threshold, logical system can detect is and connected pump to refill it. While you could certainly finish the game without ever touching these mechanics, using them, will certainly make it a lot easier. 
\newline
\newline
Factorio is a sandbox that does not aim limit you in any way, it only set ups simple rules and then you can let you imagination run wild. You do not even need to build a factory, you can set your own goal. For example, you want to build a functioning processor? You can do so, by using logical system. 
\newline
\newline
While many games of similar genre focus solely on the singleplayer experience, Factorio also supports multiplayer ( fun fact, the limit of player is theoretically 65,535). This feature offers a change of pace with the focus shifting from one-man job to teamwork, giving the players a whole different experience. The factory gets bigger, messier and most importantly the process of building it, more fun.
\newline
\newline
Bigger factories can get really messy if you do not plan ahead enough. Using top down perspective and clear prerendered graphics, Factorio manages to help you orient in messy factory quite well. This graphics style is what ties it all together. It is deceptively simple, so you lean on your imagination even more. 


\end{document}
